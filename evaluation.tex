%!TEX root = thesis.tex
\chapter{Evaluation and Results}
\label{chap:evaluation}

Table~\ref{tab:results} gives an example of how to create a table.

\begin{table}
\centering
\begin{tabular}{llrrrrrr}
  Config & Property & States & Edges & Peak & E-Time & C-Time & T-Time
  \\ \hline \hline
22-2 & A   &    7,944  &   22,419  &  6.6  \%  &  7 ms & 42.9\% &  485.7\% \\
22-2 & A   &    7,944  &   22,419  &  6.6  \%  &  7 ms & 42.9\% &  471.4\% \\
30-2 & B   &   14,672  &   41,611  &  4.9  \%  & 14 ms & 42.9\% &  464.3\% \\
30-2 & C   &   14,672  &   41,611  &  4.9  \%  & 15 ms & 40.0\% &  420.0\% \\ \hline
10-3 & D   &   24,052  &   98,671  & 19.8  \%  & 35 ms & 31.4\% &  285.7\% \\
10-3 & E   &   24,052  &   98,671  & 19.8  \%  & 35 ms & 34.3\% &  308.6\% \\
\hline \hline
\end{tabular}
\caption{Selected experimental results on the communication protocol example.}
\label{tab:results}
\end{table}