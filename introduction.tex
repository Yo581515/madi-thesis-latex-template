% !TEX root = thesis.tex
% \chapter{Introduction}
% \label{chap:introduction}

% \section{Context and Motivation}

% \section{Problem Description and Research Questions}

% \section{Research Method}

% \section{Thesis Outline}

\chapter{Introduction}\label{chap:introduction}

The seas and oceans play a key role in the operation of global ecosystems,
climate regulation, food security, and energy
production\cite{CopernicusFoodSecurity}. Monitoring of such large and complex
environments requires advanced technology. Among these, underwater sensor
networks (UWSNs) have become valuable resources for continuous real time data
collection in marine ecosystems, subsea structures, and oceanic
processes\cite{Domingos2024}.

The SFI Smart Ocean project is designed to create an autonomous and flexible
wireless marine observation program. The system will allow for large scale,
long term monitoring of underwater spaces as well as installations by combining
UWSNs and cloud based big data solutions. The infrastructure is intended to
deal with multi parameter observations, ensuring reliable data for both
scientific research as well as industrial
applications\cite{SmartOceanProject,SmartOceanSite}.

Limitations of acoustic communication, long range UWSN deployments are impacted
by environmental and technical challenges. These consist of power restrictions,
minimal data rates and the inability to recalibrate once deployed at depth.
Sensors are subjected to extreme conditions. Data quality can be affected by
pressures, biofouling, corrosion, along with electronic drift with time.
Moreover, environmental measurements are affected by parameters such as
temperature and salinity can introduce additional
uncertainty\cite{Skalvik2023}. Once data reaches surface nodes, there is a need
for efficient data handling. The storage infrastructure needs to be able to
deal with large volumes of inbound data and maintain longterm availability.

The thesis investigates techniques for managing time series data produced by
the Smart Ocean sensor network. The emphasis is on evaluating the development
of database technologies that can deal with large scale storage and effective
querying of time stamped observations. The work contributes to the wider
picture of the Smart Ocean which aims to promote data driven decision making in
marine operations. This research aligns with the ACM Computing Classification
System (CCS) under Information systems → Data management systems → Database
administration → Database performance evaluation\cite{ACM-CCS-2012}. The study
focuses on evaluating and comparing time series database technologies to
improve the efficiency and reliability of underwater sensor data management.

% \newpage
\section{Context and Motivation}
% SmartOceanSite
\subsection{The SFI Smart Ocean Platform}
The Norwegian Research Council has provided funding for the SFI Smart Ocean
project. Research institutes as well as industry partners work together on key
issues in building smart ocean systems\cite{SmartOceanHub}. The initiative
concentrates on 3 main areas:

\begin{enumerate} \item \textbf{Underwater Sensor and Measurement Technology}:
            Focuses on developing autonomous sensors and methods for real time
            monitoring
            of underwater environments. These sensors include features like
            data
            collection, acoustic communication, and energy efficient
            operations.

      \item \textbf{Underwater Wireless Sensor Networks Based on Acoustic
                  Communication}: Addresses the establishment of reliable
            communication networks
            for data transmission from underwater sensors.

      \item \textbf{Smart Ocean Platform for Cloud Based Data and Application
                  Services}: Consists of developing the Smart Ocean platform to
            incorporate,
            process, and visualize ocean information. The platform uses
            Standard APIs \&
            efficiently formats to deal with data from diverse underwater
            sensors.
\end{enumerate}

\newpage
\subsection{Motivation}
The deployment of Smart Ocean's sensor network will generate a large volume of
time series data. The underwater sensors continuously measure parameters over
extended periods, resulting in a continuous stream of timestamped readings. The
data volume rapidly reaches big data scales with dozens or possibly hundreds of
sensors reporting in real time (each measuring several variables). The system
needs to store years of data reliably, deal with high frequency ingestion, and
support queries for real time alerts and long term analysis a major challenge.

The selection of suitable data management technologies is crucial given these
requirements. Standard relational databases, although robust, are not
specifically designed to deal with high volume time series workloads where data
arrives sequentially and is mainly queried by time. In recent years,
specialized Time Series Management Systems (TSMSs) have emerged in response to
the high volume, high velocity nature of data generated by IoT devices and
sensors. Unlike general purpose databases, TSMSs are architected to efficiently
store, query, and process time stamped data streams, making them uniquely
suited for real time sensor workloads\cite{Jensen2017}. These systems are
designed to efficiently ingest and index time stamped data and provide built in
functions for time window queries, downsampling, and time centric analytics.
Dedicated TSDBs are increasingly considered a natural fit for fast sensor data
streams, yet with a wide variety of database options available, it is not
obvious which technology is most suitable for the Smart Ocean platform.

\newpage
\section{Problem Description and Research Questions}
the best way to efficiently store and manage the large amounts of time series
data generated by the Smart Ocean platform in the above mentioned context is to
determine what database technology (or combination of technologies) best meets
the platform's demands for scalability, functionality as well as dependability
in dealing with underwater sensor data. It involves comparing various
approaches. For example traditional relational databases with time series
extensions versus purpose built time series databases to figure out their
relative advantages and disadvantages for Smart Ocean's use case.

The investigation is guided by the following research question and sub
questions:

\begin{itemize}
      \item \textbf{Main Research Question:} \emph{Which database technology is
                  most suitable for managing large scale time series data
                  % from the % SFI Smart Ocean platform, and why?
            }
      \item \textbf{RQ1:} What are the key requirements and challenges in
            managing % Smart Ocean’s time series 
            sensor data (e.g., data volume, frequency of data, query patterns,
            real time access needs, and deployment constraints)?
      \item \textbf{RQ2:} Which existing database systems (relational, NoSQL,
            and dedicated time series databases) are potential candidates for
            this task, and what are their expected advantages or limitations
            % in the Smart Ocean context?
      \item \textbf{RQ3:} How do selected candidate databases perform under
            % representative Smart Ocean data workloads in terms of data
            ingestion rate, query performance, scalability, storage efficiency,
            and other relevant metrics?
\end{itemize}

\newpage
\section{Research Method}
To answer the research questions, this thesis applies the evaluation
methodology proposed by Brown and Wallnau\cite{Brown1996}. Their framework
provides a structured way to assess software technologies through the
definition of measurable benchmarks or technology deltas, that highlight how
one technology performs relative to another. The methodology is organized into
three phases: descriptive modeling, experiment design, and experiment
evaluation. Figure\ref{fig:framework} illustrates the overall structure of this
framework.

The descriptive modeling phase consists of identifying the relevant features.
This involves determining which requirements are most relevant for evaluation.
These requirements then form the basis for defining performance benchmarks such
as ingestion throughput, query latency, scalability, and storage efficiency.

The experiment design phase serves as the planning stage of the evaluation. In
this phase, hypotheses are formulated about how different database technologies
are expected to behave under representative workloads. Workload models and test
harnesses are developed to simulate large scale time series data, ensuring that
the experiments are both realistic and repeatable.

Finally, in the experiment evaluation phase, the selected databases are tested
using the defined benchmarks. The resulting measurements are collected,
analyzed, and compared to highlight the deltas between technologies. This
process enables a systematic assessment of strengths, weaknesses, and trade
offs, ultimately guiding conclusions about which database technologies are most
suitable for handling time series data at scale.

\begin{figure}
      \centering
      \includegraphics[scale=0.8]{figs/framework.png}
      \caption{Software technology evaluation framework.}\label{fig:framework}
\end{figure}

\newpage
\section{Thesis Outline}
The remainder of this thesis is organized as follows:

\begin{itemize}
      \item \textbf{Chapter 2 \- Background and Related Work:} Provides the
            necessary background and context. Reviews time series data
            characteristics and challenges, and surveys existing database
            technologies for time series data management. Discusses related
            work and prior evaluations of database performance in IoT and
            sensor data scenarios.
      \item \textbf{Chapter 3 \- Design and Analysis:} Outlines the design of
            the experimental evaluation, including the selection of database
            technologies, the design of test workloads, and the metrics used
            for assessment.
      \item \textbf{Chapter 4 \- Implementation and Prototypes:} Describes the
            implementation of the experimental setup, including the deployment
            of database systems and the creation of test environments. Details
            the development of prototypes or models used for evaluation.
      \item \textbf{Chapter 5 \- Evaluation and Results:} Presents the
            empirical results of the comparative evaluation of database
            technologies. Includes ingestion throughput, query performance,
            storage efficiency, and other findings.
      \item \textbf{Chapter 6 \- Conclusion and Future Work:} Summarizes key
            findings and contributions, provides recommendations for the Smart
            Ocean platform, and suggests directions for future work.
\end{itemize}
